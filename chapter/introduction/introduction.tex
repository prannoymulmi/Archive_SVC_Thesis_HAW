%---------------------------------------------------------------------------------------------------
% Introduction
%---------------------------------------------------------------------------------------------------
\newpage
%\part{start}
    \chapter{Introduction}
    Nowadays consumers and businesses are continuously growing aware of the value of archives. As the archived data has also proven
    to be valuable asset in the field of scientific research where an expert could archive their data for future reference. Having said that, the 
    MARS\footnote{MARS: Multi Agent Research and Simulation} framework offers an easy interactive platform for a domain expert i.e. ecologist, biologist, 
    chemist to simulate complex real life scenarios (e.g. reproduction of bacteria in a body) based on the concept of Agent-based modeling and simulation 
    (MAMS\footnote{MAMS: Multi Agent Management System}) \cite{agentModeling}.

        \section{Motivation}
        The inspiration for this work comes from the fact that an archive adds great value to the existing system by allowing the experts to be
        able to store their results and projects and use them in future. Additionally, the MARS system requires a large number of computational resources and
        an Archive service would provide an opportunity for the MARS system to improve their
        performance because this service would move some part of data from the primary storage (Ceph cluster \cite{Ceph}) to the secondary storage (NAS synology \cite{Synology}). 

        The MARS system design is based on a Microservice architecture \cite{MicroserviceNewMan} having a decentralized data \cite{atomic} access structure. 
        The MARS Resources have a hierarchal
        structure (Section \ref{subsection:MARSResource}) which has to be followed so a successful simulation can occur. As per the design of MARS these resources
        are stored separately in different storages/databases which should be accessed only by a certain API. Due to this underlying structure of the system it is quite 
        challenging to archive a given simulation with its resources. Section
        \ref{sec:problemStatement} describes those difficulties.  

        
        
        \section{Goals}
        The primary goal of this thesis is to design and implement an Archive service which would be able to deal with the decentralized
        structure of the MARS framework and provide an easy interface for users to archive and restore the projects and simulations back into the system.
       
        \par
        At the time of writing this thesis, the project resources are stored in a
        Ceph distributed file system \cite{Ceph}. It provides more efficiency, reliability, and
        scalability by separating data and metadata using a pseudo-random data distribution function\cite[p.~307]{{Ceph}} to store data in 
        a distributed system \cite{DistributedSystems}. Although
        efficiency and scalability comes at a price, it is financially more expensive to possess such a system in larger volumes.
        In contrast to its continuous data production, the primary storage volume available to the system is very limited.
        Additionally, a high level of correlation has been observed between the operating cost of a software system and its data volume. 
         Therefore, the Archive service would move the requested data from the Ceph storage to the slower NAS\footnote{NAS: Network Attached Storage} Synology
        \cite{Synology}. This cloud storage is owned by the MARS for backup and archive purpose.
           

     
    \section{Problem Statement}
        \label{sec:problemStatement}
        MARS is a complex Distributed System which brings upon different levels of 
        complexities for this thesis. The problems being dealt within the thesis are as follows:
        \begin{enumerate}
            \item
            \textbf{Data distribution:} In contrast to a monolithic application \cite[p.~94]{{softwareDesign}} where there is only one database for the whole system, 
            every Microservice in the MARS System owns a separate database which should be accessed only by itself in order to be scaled independently
            \cite[p.~27]{{Torre2017}}.
            As a result, the archive service is coupled with the other services 
            to get and post data into their respective databases creating more risk to failures.
            
            \item 
            \textbf{Data consistency and coherence:} By design the communication within the MARS system is executed via remote network call like HTTP\footnote{HTTP: Hyper Text Transfer Protocol} 
            or RPC\footnote{RPC: Remote Procedure Calls} making the transactions span over multiple systems. 
            Therefore it is extremely difficult to maintain a strong data consistency and coherence as the
            change of data in one database is unknown to the other service since an ACID \cite[p.~290]{{Haerder1983}} transaction between the databases does not exist. Thus, the archiving 
            process can unintentionally be led into a false state in case of failure during communication.

            \item 
            \textbf{Understanding MARS resource hierarchy:} The MARS Resource Hierarchy is the order in which the assets in the system have to be created. Understanding
            this complex hierarchy(Section \ref{subsection:MARSResource}) is vital sincerestoring or archiving data may lead to corrupted states.
            \item
            \textbf{Additional changes to MARS services:}
            Also, for the Archive service to behave as intended, additional functionalities have to be added into the other services. The Archive service 
            is only allowed to communicate via API gate i.e. HTTP or RPC as accessing another service's database directly is considered an Anti-pattern \cite[p.~27]{{Torre2017}}.
            It imposes another challenge in understanding the structure and algorithms of services which are drafted in various technologies.
         
        \end{enumerate}

    \newpage
    \section{Thesis Overview} 
        Here is a brief description of the thesis, providing a short overview on what each
        chapter contains.
        
        \par
        \textbf{Chapter 2: "Background:"} this chapter describes the design
        of the MARS system, the hierarchal structure of the services and the technologies used to
        run it.

        \par
        \textbf{Chapter 3: "Requirement Analysis:"} this chapter describes the functional and
        non-functional objectives of the Archive service.

        \par
        \textbf{Chapter 4: "Planning and Software Design:"} this chapter explains, the 
        methodologies for data storage, archive process and the software design.

        \par
        \textbf{Chapter 5: "Implementation:"} this chapter is where the details of the 
        implementation decision are explained.

        \par
        \textbf{Chapter 6: "Testing:"} this chapter explains in detail 
        some test cases and result validation that were carried out to give more credibility to the designed system.

        \par
        \textbf{Chapter 7: "Conclusion:"} this chapter presents the outcomes and some 
        suggestions for improvements which can be applied in the future.
        
        \textbf{Appendix}, provides a brief description of how the archive service can be built and deployed locally for future development.

