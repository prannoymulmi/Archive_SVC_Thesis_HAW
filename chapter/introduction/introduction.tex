%---------------------------------------------------------------------------------------------------
% Introduction
%---------------------------------------------------------------------------------------------------
\newpage
%\part{start}
    \chapter{Introduction}
        \section{Motivation}
            
            The MARS\footnote{MARS: Multi Agent Research and Simulation} provides a platform to simulate
            different scenarios as described by the model (Section \ref{section:MARS}). The Agent-Based Model \cite{agentModeling} simulation
            can be utilized to reproduce various real life scenarios. Since the simulations analyzes real life situations, 
            an enormous amount of agents are involved. Therefore producing substantial size of data which varies from kilobytes to terabytes
            depending on the number of agents and the complexity of the model. A very high level of correlation has been observed between 
            the operating cost of a software system and its data volume. 


            \par

            At the time of writing this thesis, the output data and the project data are stored in a
            Ceph distributed file system \cite{Ceph}. 
            
            This provides a faster and efficient method for data
            access but this storage is rather expensive to get more volume. Due to the financial cost 
            factor the primary storage volume available is very limited. Since the simulation 
            results are random in nature, it is a point of interest to keep the data for future
            research purposes with its configurations. With archiving service in hand to the system, 
            it would be possible to move the inactive or selected projects to the secondary storage. 
            The secondary storage is larger in volume and cheaper. 

            \par

            For the MARS project, this serves as enormous
            motivation to introduce a service which would archive the chosen project
            into another cheaper storage component available, in this case NAS synology 
            \cite{Synology}. Having such a service included would maintain a positive impact 
            towards the reliability and robustness of the system by having less data in the 
            active system and thus have fewer operations to be executed.
 

    \newpage        
    \section{Problem Statement}

        The main aim of this thesis is to design and test an archive service
        to the existing MARS\footnote{MARS: Multi Agent Research and Simulation} system. This service's responsibility is to move the related resources.The archived data is intended 
        to be used for future research purposes.
        The MARS is a complex distributed network having an
        hierarchal dependency with the co-existing microservices. The system being designed
        in such an architecture entails one service to be tightly coupled with one other. It is 
        a big challenge, since great amount of care should be taken into consideration of how the 
        data are being stored and the order for retrieval of the data back into the active system.
        This service will be a supporting tool for further development of the system to become more 
        efficient and compliant. 
         
    
    


    

    \section{Thesis Overview}
        Here is a brief description of the thesis, providing a short overview on what each
        chapter contains.
        
        \par
        \textbf{Chapter 2: "Theory"}, this chapter describes about the design
        of the MARS system, the hierarchal structure of the services and the technologies used to
        deploy it.

        \par
        \textbf{Chapter 3: "Requirement Analysis"}, this chapter describes the functional and
        non-functional objectives of the archive service.

        \par
        \textbf{Chapter 4: "Design Of The Solution"}, this chapter explains in detail, the 
        methodologies for data storage, archive process and the software design.

        \par
        \textbf{Chapter 5: "Implementation"}, this chapter is where the details of the 
        implementation decision are explained.

        \par
        \textbf{Chapter 6: "Result And Validation"}, this chapter explains in details about 
        some test cases and result validation that were carried out.

        \par
        \textbf{Chapter 7: "Conclusion"}, this chapter presents the outcomes and some 
        suggestions for improvements which can be applied.
        
        \par
        \textbf{Appendix}
        MUST BE WRITTEN

