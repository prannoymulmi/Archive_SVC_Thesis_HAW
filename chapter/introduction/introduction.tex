%---------------------------------------------------------------------------------------------------
% Introduction
%---------------------------------------------------------------------------------------------------
\newpage
%\part{start}
    \chapter{Introduction}
        \section{Motivation}
            
            The MARS\footnote{MARS: Multi Agent Research and Simulation} provides a platform to simulate
            different scenarios as described by the model (Section \ref{section:MARS}). The Agent-Based Model \cite{agentModeling} simulation
            can be utilized to reproduce various real life scenarios. Since the simulations evaluates real life situations, 
            an enormous amount of agents are involved. Therefore producing substantial size of data which varies from kilobytes to terabytes
            depending on the number of agents and the complexity of the model. 

            \par

            At the time of writing this thesis, the output data and the project data are stored in a
            Ceph distributed file system \cite{Ceph}. It provides more efficiency, reliability, and
            scalability by separating data and metadata using a pseudo-random data distribution function\cite[p.~307]{{Ceph}} to store data in 
            an Distributed System \cite{DistributedSystems}. Although
            efficiency and scalability comes at a price, it is financially more expensive to posses such a system in larger volumes.
            In contrast to its continuous data production, the primary storage volume available to the system is very limited.
            A very high level of correlation has been observed between the operating cost of a software system and its data volume. 
            Also, it is a point of interest to archive the result data and its resources for future analysis and research purposes. In addition,
            the MARS system also possess a slower but cheaper storage component called synology 
            \cite{Synology}.

            \par

            For the system, this serves as a big
            motivation to introduce a service which would archive  project resources with its
            simulation results in the synology drive. This service being available would provide a user a possibility
            to move inactive data away from the more expensive Ceph storage to the cheaper storage volume.
            It can be considered as a promising approach towards maintaining a reliable and robust system by not having unrequired data in the 
            active system. As a result, the active system can be released with additional strain of inactive data consuming valuable resources.

 

    \newpage        
    \section{Problem Statement}

        The main aim of this thesis is to design, implement, and test an Archive service
        to the existing MARS\footnote{MARS: Multi Agent Research and Simulation} system. This service's responsibility is to archive a project's resources,
        its simulation results, and restore it back to the active system on request. 
        
        The MARS is a complex Distributed System designed using an Microservice \cite{MicroserviceNewMan} architecture. This brings upon different levels of 
        complexities for this thesis. The problems being dealt within the thesis are as follows:
        \begin{enumerate}
            \item
            \textbf{Data Distribution}, In contrast to a monolithic application \cite[p.~94]{{softwareDesign}} where there is only one database for the whole system, 
            every Microservice in the MARS System owns its personal database which should be accessed only by itself in order to be scale independently
            \cite[p.~27]{{Torre2017}}.
            As a result, the archive service is coupled with the other services 
            to get and post data into their respective databases creating more risk to failures.
            
            \item 
            \textbf{Data Consistency And Coherence}, The design of MARS system for communication is only via remote network call like HTTP\footnote{HTTP: Hyper Text Transfer Protocol} 
            or RPC\footnote{RPC: Remote Procedure Calls}. This makes it extremely difficult to maintain a strong data consistency and coherence as the
            change of data in one database is unknown to the other service since an ACID transaction does not exist, which can lead to archive of false data.

            \item
            \textbf{},
            Also, for the archive to function as expected some changes have to be made in the other Microservices in the system which are written in different
            programming languages having their own level of complexities. 

            \item
            \textbf{Operational Complexities},The system in addition also has operational complexities due to the fact a lot of services have to be managed and it gets very hard to run all the services
            in a local environment to run tests i.e. integration, end-to-end tests.    
            
        \end{enumerate}

    \newpage
    \section{Thesis Overview} 
        Here is a brief description of the thesis, providing a short overview on what each
        chapter contains.
        
        \par
        \textbf{Chapter 2: "Background"}, this chapter describes about the design
        of the MARS system, the hierarchal structure of the services and the technologies used to
        deploy it.

        \par
        \textbf{Chapter 3: "Requirement Analysis"}, this chapter describes the functional and
        non-functional objectives of the archive service.

        \par
        \textbf{Chapter 4: "Design Of The Solution"}, this chapter explains in detail, the 
        methodologies for data storage, archive process and the software design.

        \par
        \textbf{Chapter 5: "Implementation"}, this chapter is where the details of the 
        implementation decision are explained.

        \par
        \textbf{Chapter 6: "Result And Validation"}, this chapter explains in details about 
        some test cases and result validation that were carried out.

        \par
        \textbf{Chapter 7: "Conclusion"}, this chapter presents the outcomes and some 
        suggestions for improvements which can be applied.
        
        \par
        \textbf{Appendix}
        MUST BE WRITTEN

