%---------------------------------------------------------------------------------------------------
% Analysis
%---------------------------------------------------------------------------------------------------
\newpage
\chapter{Requirement Analysis}
\label{chap:ReqAnalysis}
 The main requirement of this thesis is to design, and implement an Archive service i.e. back-end web service for the MARS framework. The service's role is to
 archive the MARS resources mentioned in Subsection \ref{subsection:MARSResource} from the Ceph cluster \cite{Ceph} to the 
 NAS\footnote{NAS: Network Attached Storage} Synology drive \cite{Synology}.

 This service targets any user who desires to archive the MARS resources needed for a simulation including the existing simulation results, which would
 be analyzed in the future. The Archive service would expose its API, calling it, one can archive and restore the resources. The exposed
 API can later be integrated in the MARS graphical interface (MARS Teaching User Interface). The MARS Teaching API acts as proxy between the users 
 and the MARS back-end Microservices, as it provides some level of abstraction by hiding unnecessary endpoints for the user in the graphical interface.

 Figure \ref{fig:archiveUseCase} presents the use case of the system with its expected behavior. The main intention of the diagram is to show the
 general description of the possible actions using this service. As seen from the diagram, a user is associated with three use cases i.e. archive, restore,
 and getting the job status. 

 Following the use case diagram of the system, a step of marking the project is done before archiving the project or else the process fails. This is desired because 
 during the archive, adding new or editing the resources may lead to incomplete/wrong data being copied to the Synology. To clarify, the case of new resources being
 added or being changed is possible due to the fact that, a project is accessible to more than one user at any given time. The current structure of the system
 poses data synchronization issues. Therefore, if the marking fails the archive project will not occur, leading the archiving use case being extended from the marking
 use case. The second action which can be triggered by the user is to retrieve the archived project from the Synology back to active system. Lastly, since archiving 
 and retrieving are could take from minutes to hours depending on the size of the project, these processes are visioned to be a background long running job. For the user
 it would be a point of interest to know if the job is completed or not. To fulfill this purpose, the use case Get job status is present in the system.
 
 \begin{figure}[H]
    \centering \includegraphics[scale=0.6, angle=90, origin=c]{grafiken/archiveUseCase.png}
    \caption{Use case diagram for Archive service}
    \label{fig:archiveUseCase}
\end{figure}
 
Different uses cases have been mentioned for the archive service in Figure \ref{fig:archiveUseCase}. Accomplishing them requires several requirements to be met
which are divided into two different categories technical and, functional aspects. 

\section{Technical Requirements}
\label{section:technicalReq}
The requirements specified in this section present us the technical aspects of the Archive service. A tabular description 
(Table \ref{table: Technical Requirements}) is presented below.
The detailed description depicts the benchmarks of how the system should be designed to meet the needs for a better sustainable prototype.
The result from this work delivered must comply with the following technical requirements.

    \begin{table}[h!]
        \centering
        \begin{tabular}{|p{3cm}|p{12cm}|}
            \hline
                \textbf{Requirement}  & \textbf{Description}\\
            \hline
                 Build and deployment & 
                 The service should be deployable in the MARS kubernetus \cite{kubernetes} cluster using the Gitlab pipeline as seen in Figure \ref{fig:CIbuild}
                 which is valid at the time of writing this Thesis. In addition,
                 the build stages for the pipeline also has to be written. \\
            \hline
                Docker containerizing.
                & Choose a suitable Docker container \cite[p.~7 - 8]{Torre2017} for the service. This depends on the type of programming language which 
                the service is being
                written in (e.g. dotnet core, python, java, go).\\
            \hline
                 Scalability & Provide further extensibility option using object orientation so that the service can be scalable.\\
            \hline
                 Robustness & The system should be tested against multiple cases to ensure correct functionality of the service using unit tests and
                 integration tests.\\    
            \hline
                 Connection to Synology & The Archive service should be able to make a connection to the Synology drive and store desired data successfully.\\    
            \hline
                Follow Microservice patterns & The service should follow data sovereignty pattern for Microservices mentioned in 
                (Subsection \ref{subsection:dataSovereignty}).\\ 
            \hline
                Responsiveness & The API should give some kind of feedback to the user never the less if request cannot be made an 
                error message should be returned instead of no result. \\      
            \hline
        \end{tabular}
        \caption{Technical requirements for the Archive service}
        \label{table: Technical Requirements}     
    \end{table}    
  
    \begin{figure}[H]
        \centering \includegraphics[scale=0.5]{grafiken/CIbuild.png}
        \caption{MARS Continuous Integration Pipeline build}
        \label{fig:CIbuild}
    \end{figure}

    Figure \ref{fig:CIbuild} presents the Continuous Integration system which is being followed at the time of writing this thesis by the MARS developer community.
    The CI\footnote{CI: Continuous Integration} pipeline would be triggered as soon as a new commit is being pushed to the remote 
    GitLab\footnote{https://about.gitlab.com} repository. This would then build the docker image of the service with the new changes. The next step would be to
    run the unit tests written for the service, which is a mandatory step. Lastly, if the pipeline passes, the docker image will be pushed
    to the GitLab registry\footnote{https://about.gitlab.com/2016/05/23/gitlab-container-registry/}. If this is successful then the image can be used in one of 
    the MARS kubernetes cluster i.e. MARS beta, MARS production environments. In addition,
    it can be seen that the integration tests are configured to be an optional test because the integration tests could consume considerably more time and may hinder
    important updates. For this reason, the integration tests are ran manually in the pipeline.

\section{Functional Requirements}
    \label{section:functionalReq}
    This section describes the functional requirements for the Archive service.
    The functional aspects which carves the Archive service are mentioned below.

    \subsection{Archive Project Resources}
        The designed system must be able to archive the MARS resources from the active system (Ceph cluster at the time of writing) 
        into the Synology \cite{Synology}. The application must
        be able to archive a project which does not have all the resources mentioned in Table \ref{table: archivedMars} (e.g. no simulation runs have been triggered).
        This must be supported since it could be the case that the user wants to archive only some of the resources.
        
        \begin{table}[H]
            \centering
            \begin{tabular}{|p{3cm}|p{12cm}|}
                \hline
                    \textbf{Resource Name}  & \textbf{Description}\\
                \hline
                     Metadatas & 
                     This resource stores the metadata (e.g. file id, file name). It gives the system the information about
                     existing files in the system. \\
                \hline
                     Files & 
                     This resource correspond to the models (e.g. wolves and sheep model) and input files (e.g. GIS, Time series) which describe a simulation. \\
                \hline
                     Scenarios & 
                     This resource defines the parameters for the model which would be simulated (e.g. simulation run time, number of agents). \\
                \hline
                     Result Configurations & 
                     This resource defines which parameters of a model and layers are going to be stored in the database that will be used for visualization.\\
                \hline
                     Simulation Plans & 
                     This resource contains the scenario and the result configuration which can be executed. The simulation plan could be configured to
                     have different scenario and result configuration to produce different kind of output.\\
                \hline
                     Simulation Runs & 
                     This resource contains the metadata for a simulation results i.e. simulation id, simulation status.\\
                \hline
                     Simulation Results & 
                     This resource is the output and contains the results for a single simulation run.\\
                \hline
            \end{tabular}
            \caption{MARS resources which are to be archived}
            \label{table: archivedMars}     
        \end{table} 
        
        \subsubsection{Assurance of correct data being persisted}
            MARS being a distributed system, data coherency (Subsection \ref{subsection: distriChallenges}) 
            is one of the big issue which this thesis faces. As a consequence,
            wrong or unwanted data could be archived. Therefore, the Archive service must ensure that while an archive is running the data is not
            altered.
        
        \subsection{Retrieve Project Resources}    
            \label{ssec:retrieveAnalysis}
            The designed software must support the retrieval of the archived projects from the Synology into the active system. The
            system must be able to restore the project given that, the services support the data format which is archived in the Synology.
           
            \begin{figure}[H]
                \centering \includegraphics[scale=0.4]{grafiken/synology.png}
                \caption{Archive service's communication structure}
                \label{fig:synology}
            \end{figure}

            Figure \ref{fig:synology} illustrates that the Archive service must only use the volume assigned for archiving and nothing more. 
            This requirement must be fulfilled to comply with the MARS development standard. It also has to be made sure that the retrieved
            resources are usable (e.g. the restored simulation plans should be able to run a simulation again). 

        \subsection{Archive and Retrieve Process Status}  
            The archive and retrieve processes are long running tasks. The designed software must run these processes in the background to avoid 
            long waiting time for another requests. Given this, an API endpoint must be made available which gives the current status of the archive 
            or retrieve job. Using the status a user can determine whether the job is being executed or is finished.


        \subsection{Download Archived Data as a Compressed File}
        \label{sec:anaCompress}
            It is of great importance for a domain Expert i.e. ecologist who are not technical experts to have a graphical interface. In this interface,
            it must be possible to navigate to the project of interest and easily download the project as a zip file. There could be cases where the
            MARS system is out of order and the data is required. Then it can be accessible by anyone with basic knowledge of the system.
        
        \subsection{Fault-Tolerant Design}   
        Firstly, the Archive service has to communicate with many services in the system, leading the rate of failure being higher
        in comparison to a system which does not depend on other services. A breakdown 
        of one service would cause the whole process of archive/retrieve to stop unexpectedly. Secondly, it is also possible that a running Archive service can be terminated
        due to some unexpected reason. Therefore, fault tolerance mechanism has to be included in the archive system, so that it has a chance of 
        recovery. 
