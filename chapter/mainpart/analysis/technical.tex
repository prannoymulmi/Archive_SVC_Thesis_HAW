\section{Non Functional Requirements}
\label{section:technicalReq}
The requirements specified in this section present us the technical/non-functional aspects of the Archive service. A tabular description 
(Table \ref{table: Technical Requirements}) is presented below.
The detailed description depicts the benchmarks of how the system should be designed to meet the needs for a better sustainable prototype.
The result from this work delivered must comply with the following technical requirements.
\begin{table}[h!]
    \centering
    \begin{tabular}{|p{3cm}|p{12cm}|}
            \hline
                \textbf{Requirement}  & \textbf{Description}\\
            \hline
                 Build and deployment & 
                 The service should be deployable in the MARS Kubernetus \cite{kubernetes} cluster using the Gitlab\footnote{https://about.gitlab.com} pipeline as seen in Figure \ref{fig:CIbuild}
                 which is valid at the time of writing this Thesis. Also,
                 the build stages for the pipeline also has to be written. \\
            \hline
                 Extensibility & The system must be made extensible so that future requirements can be easily added.\\
            \hline
                 Robustness & The system must be able to cope with different kinds of errors during execution.\\    
            %\hline
             %    Performance &  The system must be able to archive a file of 500 MB within a minute (8.3 MB/s).\\    
            \hline
                 Logging &  The service must provide logging information.\\    
            \hline
                 Usability & The system should be integrated into the MARS UI so that it is easily usable by all end users.\\     
            \hline
                 Make a Swagger API interface & The Archive service should have a Swagger \cite{swagger} interface available so that other 
                 developers can use the service with ease.\\         
            \hline
                Follow Microservice patterns & The service should follow the data sovereignty pattern for Microservices mentioned in 
                (Subsection \ref{subsection:dataSovereignty}).\\ 
            \hline
                Responsiveness & The API should give some feedback to the user never the less if the request cannot be made an 
                error message should be returned instead of no result.\\      
            \hline
           
    \end{tabular} 
    \caption{Technical requirements for the Archive service}
    \label{table: Technical Requirements}  
    \end{table} 
  
    