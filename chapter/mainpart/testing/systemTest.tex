\section{System Test}
This test verifies the requirements mentioned in Section \ref{section:functionalReq} by performing manual GUI tests. System test are manual tests
which ensures the correct behavior of the system.

\subsection{Successful Archive Process Start}
For this test, it is verified that the archive process can be started from the MARS Teaching GUI with the precondition of no other process
for this project is running. One Wolves and Sheep model is uploaded including the resources i.e. scenario, result configuration,
simulation plan, simulation run, simulation result. As this test was to verify the successful execution of the process, only one of each resources
were uploaded and checked.

\subsection{Archive with a More Complex Model}
For this test, the Kruger National Park (KNP) model is used instead of the Wolves and Sheep because the KNP is more complex since it requires more layers
i.e. GIS, Timeseries, Geopotential layers. The successful archive of this model with all its resources i.e. scenario, result configuration, simulation plan, simulation run,
simulation results were verified. 

\subsection{Successful Data Archive in Synology}
For this test, the Synology drive where the archives were supposed to be uploaded were verified. The successful archive process of the Wolves and Sheep model
was correctly stored in the archive folder inside the drive.

\subsection{Successful Retrieve Process Start}
For this test, it is verified if the retrieve process could be successfully started from the MARS Teaching GUI when no process for the project is running.
Both the KNP and Wolves and Sheep models with its resources were restored back to the system. 

\subsection{Correctness of the Restored Project}
For this test, it is verified if the retrieved data are the same as for the archive. Also, a check is done whether the restored files can be used in the active system
to produce further results (e.g. a simulation was ran from the restored simulation plan, a new scenario is created from the model).

\subsection{Fault-Tolerance Test}
For this test, a intentional error is created to verify if the implemented strategy for fault tolerance is executed. A successful verification for this
strategy was checked. Also a check was done by removing the server from the cluster while a job was being processed. This checked and verified that the Archive service
would restart the job in case of sudden failure after the server has be revived. This is the case only if the defined number of retires are not exceeded.