\chapter{Testing}
This chapter presents the methodology used for testing the correct functionality of the application. The validation consist of set of automatic test i.e.
Unit tests and Integration test including a System test to ensure the proper functionality of the service. The verification is done on both the functional
aspects of the back end logic and also on the GUI controls for the Archive service.

\begin{figure}[H]
    \centering \includegraphics[scale=0.5]{grafiken/CIbuild.png}
    \caption{MARS Continuous Integration Pipeline build}
    \label{fig:CIbuild}
\end{figure}

Figure \ref{fig:CIbuild} presents the Continuous Integration system which is being followed by the MARS developer community. This pipeline plays an important role
for the maintenance of the service because the automatic tests are executed here.
The CI\footnote{CI: Continuous Integration} pipeline would be triggered as soon as a new commit is being pushed to the remote 
GitLab\footnote{https://about.gitlab.com} repository. This would then build the docker image of the service with the new changes. The next step would be to
run the unit tests written for the service, which is a mandatory step. Lastly, if the pipeline passes, the docker image will be pushed
to the GitLab registry\footnote{https://about.gitlab.com/2016/05/23/gitlab-container-registry/}. If this is successful then the image can be used in one of 
the MARS kubernetes cluster i.e. MARS beta, MARS production environments. In addition,
it can be seen that the integration tests are configured to be an optional test because the integration tests could consume considerably more time and may hinder
important updates. For this reason, the integration tests are ran manually in the pipeline.

\section{Unit testing}
These tests are designed to verify 
\section{Integration Test}
The main aim of this test is to test the communication between the different services that the Archive service is dependent upon. This test is designed in 
such a way that it calls all the endpoints that is used in order to create a simulation run from adding files to running the simulation. As a result of checking
these endpoints it gives an additional benefit of detecting errors introduced from other services in the area of resource creation. As an example, assuming a service
(excluding the Project, User, Marking, and Deletion service)
made some changes which has some bug with resource creation then if the Archive service integration test are executed it would try to recreate the mock resource
and would result in a fail which could then be detected.

\subsection{Challenges}
The Integration Tests are very beneficial to have a more stable system but realizing this test posed really big challenges which took considerably long time with
some constraints. Due to the Microservice architecture of the MARS framework the services are deployed as an independent entity which have their own database.
It was an enormous task to figure out how to combine all these independent services and have them running in a testing environment. The solution to this issue
was to create a multi-container Docker application using Docker-compose. Here, the images of the services required are loaded and the Mongo db database for each
service is seeded. It is also very important for the order for the services and the seeding to loaded in a specific order since the services are dependent upon
each other.  

\begin{lstlisting}[language=docker-compose,caption={Docker compose configuration snippet for Archive service Integration Test}, captionpos=b, breaklines=true,label={code:integCompose}]
archive_svc_tests:
    image: nexus.informatik.haw-hamburg.de/microsoft/dotnet:2.0.0-sdk
    volumes:
      - ../:/mars-archive-svc
    entrypoint:
      - sh
    command:
      - ./mars-archive-svc/IntegrationTests/run_tests.sh
    links:
      - mongodb
      - metadata-svc
      - scenario-svc
      - file-svc
      - reflection-svc
      - resultcfg-svc
      - sim-runner-svc
      - sim-runner
      - mongo-seed
      - result-mongodb
      - result-mongo-seed
      - database-utility-svc
      - marking-svc
    depends_on:
      - mongodb
      - metadata-svc
      - scenario-svc
      - file-svc
      - reflection-svc
      - resultcfg-svc
      - sim-runner-svc
      - sim-runner
      - mongo-seed
      - result-mongodb
      - result-mongo-seed
      - database-utility-svc
      - marking-svc
\end{lstlisting}      

Listing~\ref{code:integCompose} shows a snippet of the docker compose file for running the Integration Test. The depends\_on attribute for the archive service
has many services in it. This means the archive\_svc\_test is waiting for the other services to load and the mongodb to be seeded with mock data. Unfortunately,
since the Project service does not use Mongodb but instead Postgres Sql as a database for some unknown reason the synchronization of the seeding of the data
did not succeed and therefore the tests could not be ran. The
Marking service and Deletion service endpoints that the Archive service uses have a dependency with the Project service  which resulted as a unsuccessful test
for them. Although, it is a point of interest in the future to investigate to figure out the reason for this and complete the whole test.

\subsection{Correctness of received data}
For this test, the GET endpoints of the services were tested. A dummy model for the data which the Archive service expects is compared to the result
form the GET endpoints. This check would aid to verify if the services return a data model which the service. If any changes to the other services
related to the data model is made this test would detect it. 

\subsection{Correctness for uploading data}
For this test, the POST, PUT and PATCH endpoints were tested. It is designed to verify during the restoring process if the expected data model 
is still compatible with the services or if there is some error introduced after a new change. The process is conducted by uploading the 
data models using the mentioned endpoints and in return expecting a success status.

\subsection{Correctness of response}
This test is designed to check the interface of the API provided by the other services. This check would aid to validate if the API of the
service returns a correct status as described by their Swagger interface. As an example, while posting a scenario in the scenario service if the
name and the data id of the model is already existing an conflict status code will be returned by the request. Therefore, for this example an existing
model would be uploaded and the test would verify if an conflict status is returned. Likewise, the other responses are verified by this test. 

\subsection{Integration with the Database}
This test is designed to test if a proper integration between the Mongodb and Archive service is maintained. The Archive service stores vital metadata
which gives the client information about the status of the job and many other data that it needs. This check verifies the correctness of the read and write 
operations done with the Database. 

\subsection{Test Coverage}
A total of 81 tests were made which would aid in verifying the integration of the Archive service with the dependent services and the database. The 
total time for running the test measured at the time of writing the thesis is on average 7m38s. Table \ref{table:MARS Resource Hierarchy Service Overview}
mentions the services that the archive service has dependency towards and 6 out of 9 services are being tested for the integration test. Among the total services 
which were supposed to be tested only 67\%  were carried out. It is a point of interest in the future to complete these requirements which would boost the
reliability of the Archive service. 

\section{System Test}
This test verifies the requirements mentioned in Section \ref{section:functionalReq} by performing manual GUI tests. System test are manual tests
which ensures the correct behavior of the system.

\subsection{Successful archive process start}
For this test, it is verified that the archive process can be started from the MARS Teaching GUI with the precondition of no other process
for this project is running. One Wolves and Sheep model is uploaded with other resources i.e. scenario, result configuration,
simulation plan, simulation run, simulation result. As this test was to verify the successful execution of the process, only one of each resources
were uploaded and checked.

\subsection{Archive with a More Complex Model}
For this test, the Kruger National Park (KNP) model is used instead of the Wolves and Sheep because the KNP is more complex since it requires more layers
i.e. GIS, Timeseries, Geopotential layers. The successful archive of this model with all its resources i.e. scenario, result configuration, simulation plan, simulation run,
simulation results were verified. 

\subsection{Successful Data Archive in Synology}
For this test, the Synology drive where the archives were supposed to be uploaded were verified. The successful archive process of the Wolves and Sheep model
was correctly stored in the archive folder inside the drive.

\subsection{Successful Retrieve Process Start}
For this test, it is verified if the retrieve process could be successfully started from the MARS Teaching GUI when no process for the project is running.
Both the KNP and Wolves and Sheep models with its resources were restored back to the system with all its resources. 

\subsection{Correctness of the Restored Project}
For this test, it is verified if the retrieved data are the same as for the archive. Also, a check is done whether the restored files can be used in the active system
to produce further results (e.g. a simulation was ran from the restored simulation plan, a new scenario is created from the model).

\subsection{Fault-Tolerance Test}
For this test, a intentional error is created to verify if the implemented strategy for fault tolerance is executed. A successful verification for this
strategy was checked. Also a check was done by removing the server from the cluster while a job was being processed. This checked and verified that the Archive service
would restart the job in case of sudden failure after the server has be revived. This is the case only if the defined number of retires are not exceeded.
\section{Performance Test}
This test verifies the performance metrics of the Archive service by analyzing different numbers of files, file sizes, and archive strategies.

\subsection{Archive Performance}
For this test, an archive process is executed and repeated 4 times to get an average general performance overview. Figure \ref{fig:archivePerformance} illustrates a bar diagram for running an archive process (zipped simulation results) with 7 files, 2 scenarios, 2 result configurations, 2 simulation plans,
12 simulation runs, and results. The results seen in the figure shows that it took about 34.9s on average to run this process. It is also to be noted that the
simulation results are zipped making their total size about 86 Mb. Figure \ref{fig:archivePerformanceUn} shows the archive process of the same project without 
compressing the simulation results. It can be seen that the file sizes for the uncompressed process are significantly higher in contrast to the compressed
process with file size on avg about 683 Mb and 43s of processing time.

\begin{figure}[H]
    \centering \includegraphics[scale=0.45]{grafiken/archiveZip.png}
    \caption{Overall performance of the archive process (compressed simulation results)}
    \label{fig:archivePerformance}
\end{figure}

\begin{figure}[H]
    \centering \includegraphics[scale=0.45]{grafiken/archiveUnzip.png}
    \caption{Overall performance of the archive process (uncompressed simulation results)}
    \label{fig:archivePerformanceUn}
\end{figure}

Unfortunately, simulation results with larger file sizes could not be tested as no such simulation results were available at the moment. However, with this metrics, it can be said that
for smaller file sizes, it is better to compress the simulation data while archiving it since there is a significant volume save and the time taken for the
archive does not differ by much. It is also seen that the I/O is a bottleneck because the larger file (uncompressed) needed more time to archive than the smaller (compressed) file. 

\subsection{Retrieve Performance}
For this test, the same project that was archived is restored to analyze the performance of the retrieve process. Figure \ref{fig:restorePerformance} illustrates
the results of the retrieve process with the compressed simulation results which takes 4.3 mins on average. Figure \ref{fig:restorePerformanceUn} illustrates the
results of the retrieve process using the uncompressed simulation results which took on average of 4 mins. 

\begin{figure}[H]
    \centering \includegraphics[scale=0.5]{grafiken/retrieveZip.png}
    \caption{Overall performance of the retrieve process (compressed simulation results)}
    \label{fig:restorePerformance}
\end{figure}

\begin{figure}[H]
    \centering \includegraphics[scale=0.5]{grafiken/retrieveUnzip.png}
    \caption{Overall performance of the retrieve process (uncompressed simulation results)}
    \label{fig:restorePerformanceUn}
\end{figure}

Also, to verify the complexity of the processes, the number of simulations were doubled, i.e., 24 simulation and the performance was recorded
and an average of 55 sec for archive and 8.2 mins for retrieve was observed. As a result, the processing time was doubled as expected.