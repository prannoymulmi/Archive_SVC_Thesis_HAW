\newpage
\section{Archive}

Archiving in computer science is an act of storing single or collection of computer files with its metadata for long-term retention. The data being archived is 
not needed currently in the active system. Generally these data are valuable for an organization or an individual not to be discarded but are needed seldom. 
Bringing forward the need of relocating the data into a cheaper storage i.e. archiving. Traditionally, these kinds of data were stored in magnetic tapes but nowadays 
due to availability of cheaper NAS\footnote{NAS: Network Attachted Storage} primary storage, 
these storage means are being preferred. 
The advantages of using NAS i.e. Synology over magnetic tapes \cite{Synology}.
\begin{enumerate}
    \item \textbf{File Sharing} It is easier for many user to access the archived data because the storage is connected to the cloud. Whoever has access to the
    network can get the data compared to magnetic tapes where one needs hold of the tape in order to get the required data.
    \item \textbf{Better Security} The NAS provides a better security since its can be configured to be used in a private cloud. This would prevent unwanted users
    to get hold of the data.
    \item \textbf{Easy Usability} The system is easy to manage since the system provides with easy installment, and also providing a graphical interface for file access.
\end{enumerate}
Often an archive can be confused
with a backup of a system. The key differences for archive and backup are mentioned in Table \ref{table:archiveVsBackup}.

\begin{table}[h!]
    \centering
    \begin{tabular}{|p{8cm}|p{8cm}|}
        \hline
            \textbf{Archive}  & \textbf{Backup}\\
        \hline
            It is unused but a desired copy of the data useful for future use.& 
            It is a copy of the current active data store used to recover from data corruption. \\
        \hline
            The data is relocated from the current storage system onto a less expensive storage.
            & The data is just a copy of the working copy and may or may not be stored in the same storage as the active system.\\
        \hline
             The duration for keeping an archive is longer since it would in most cases not change frequently.
             & The duration of the backup would be less compared to archiving since it would be updated frequently (e.g. daily, weekly, monthly) to have the newest 
             working copy.\\
        \hline
    \end{tabular}
    \caption{Differences between archive and backup}
    \label{table:archiveVsBackup}     
\end{table}    