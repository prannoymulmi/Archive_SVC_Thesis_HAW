\newpage
\section{Fault tolerance and maintenance implementation}
The MARS system is very susceptible to different kinds of errors due to the architectural factors mentioned in the previous chapters. Therefore, a fault tolerant
system is needed which would try to recover in case of any transient errors. For the Archive service after some research an open source framework
named Hangfire\footnote{https://www.hangfire.io} seemed to be a good choice. This framework allows one to make easily repeat a a background job if an failure
occurs. It is backed up with a persistent storage where the job parameters are stored which has very easy to configure parameters. Listing \ref{lst:hangfire}
shows the simplicity of how a job retry can be configured in case of an exception is made. 

\begin{lstlisting}[language={[Sharp]C}, caption={Hangfire retry attempt configuration}, captionpos=b,label={lst:hangfire}]
private IServiceCollection BuildServices(IServiceCollection services)
{
    // hangfire job filter
    GlobalJobFilters.Filters.Add(new AutomaticRetryAttribute { Attempts = 4 });
}
\end{lstlisting}

\subsubsection{AOP Logging}
Logging is very important part of the service as it gives the maintainers key information of what has happened to the system. This makes logging important
in each part of the system which leads to copying of the logging logic to the different components in the archive service violating the SOLID principles. 
To solve this issue the interceptor pattern is used in this here. Using this pattern a decorator would be put in the class which wants to have the common
logging logic. Listing \ref{lst:interceptor} shows the implementation of the decorator "Intercept("logger")" which is injected in the class. The 
classes decorated with attribute would have a logging logic triggered. Listing \ref{lst:interceptor} shows the main logic for the logging where a success message
with all the method arugments, time elapsed would be printed in case of a successful run and will log the exception in case of failure. This aids in the maintenance
of the system as new feature are added only the decorator must be added to log the start, success or exception of the method.

\newpage
\begin{lstlisting}[language={[Sharp]C}, caption={Interceptor decorator example}, captionpos=b,label={lst:interceptor}]
[Intercept("logger")]
public class ArchiveFile: IArchiveFile
{
    public ArchiveFile(IFileClient fileClient)
    {
        _fileClient = fileClient;
    }
}
\end{lstlisting}            

\begin{lstlisting}[language={[Sharp]C}, caption={Interceptor logger logic implementation}, captionpos=b,label={lst:interceptor}]
private async Task InterceptAsync(IInvocation invocation)
{
//Before method execution
var stopwatch = Stopwatch.StartNew();

try
{
    //Calling the actual method, but execution has not been finished yet
    invocation.Proceed();

    //Wait task execution and modify return value
    if (invocation.Method.ReturnType == typeof(Task))
    {
        try
        {
            await InternalAsyncHelper.AwaitTaskWithPostActionAndFinally(
                (Task) invocation.ReturnValue,
                exception =>
                {
                    if (exception == null)
                    {
                        Console.WriteLine("Success: Class Name: {0} Method: {1}  Duration: {2:0.000} ms, Arugments: {3}", GetClassName(invocation),
                            invocation.MethodInvocationTarget.Name, stopwatch.Elapsed.TotalMilliseconds, GetMethodArguments(invocation));
                    }
                    else
                    {
                        Console.WriteLine(exception);
                        throw exception;
                    }
                    
                });
        }
        catch (Exception)
        {
            Console.WriteLine("Exception occured in async call");
        }
        
    }
}
\end{lstlisting}            
