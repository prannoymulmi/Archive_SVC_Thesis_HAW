\section{Retrieve Process Design}
This section describes the design and behaviour for restoring the archived project from the Synology back to the system
so that it can be used again using the MARS UI.   

\subsection{Decision of File Upload via File-svc}

\begin{figure}[H]
    \centering \includegraphics[scale=0.4]{grafiken/mars-cloud.png}
    \caption{File Upload in MARS Cloud \cite{MARSCLoud}}
    \label{fig:MARSCloud}
\end{figure}

Figure \ref{fig:MARSCloud} illustrates how a file upload in the MARS cloud is done at the time of writing this thesis. 
The MARS cloud is a complex Distributed System consisting of many 
Microservices and databases at its disposal. It is a point of interest how the project would be restored back as there are different possibilities for it. 
Although, it is a requirement for the archive service
to call the corresponding service to access add or modify the resources (Chapter \ref{chap:ReqAnalysis}). As mentioned in earlier chapters the MARS system
has different types of files (e.g. models, timeseries, GIS) which are managed by their own service. The files can be uploaded in two different methods.
\begin{enumerate}
 \item \textbf{Upload files via File-svc} The File-svc is a service which takes in all kind of inputs GIS, models, timeseries and it communicates to the
 concerning service by checking the fle type. This is the only method which is possible in the UI.  
 \item \textbf{Upload the respective file via its service} This method requires the Archive service to communicate with each service of the file type. 
The archive service can communicate to any service provided an endpoint. Therefore, it is also possible to upload the different kinds of files using the
corresponding service instead of the file service. If a file type model is to be uploaded a API call to the reflection service has to made. 
\end{enumerate} 
The File-svc can be seen as an abstraction layer for uploading different types of files. This layer reduces the number of direct dependencies to the Archive
service because it does not call the other services directly. Choosing the File-svc also provides an additional advantage if a new file type is added in the future.
In this case the archive service does not need to modify any code to upload the new file type. Given the reasons to have cohesion and easy maintenance, file uploads
via File-svc deemed to be a better choice. 

\subsection{Retrieve as an Atomic Action}
\begin{figure}[H]
    \centering \includegraphics[scale=0.45]{grafiken/restoreActivity.png}
    \caption{Activity Diagram for retrieving a project}
    \label{fig:activityRestore}
\end{figure}

Figure \ref{fig:activityRestore} depicts the activity diagram for restoring a project. The retrieving process is also a background job due to the same reason
as for the archive i.e. long running times. The first step after creating the retrieve job is to get the metadata from the Synology and then upload all the files.
All the files have to be finished uploading and processed, otherwise the sequential steps would not have references of the files. After all the uploads are complete
the scenarios gets the reference to the file id so that it could uploaded. Following the scenarios, the result configurations are also uploaded for the corresponding models.
As the simulation plans is dependent upon the scenarios and the result configurations this is the next resource which will be uploaded. Lastly, the simulation runs
and the simulation results would be uploaded respectively. 

In case an error occurs a Two-phase commit protocol \cite{atomic} is adopted. This strategy is taken into consideration to bring atomicity on decentralized data
as it tries to roll back if the distributed transaction fails.
Due to chances of failure, an incomplete data restore process could occur. In the MARS system one cannot work with having
incomplete data since the resources are dependent upon each other. Having an atomic transaction for the retrieve process would be a simple mechanism to overcome
this issue. In case of any failure during retrieval, the partially restored resources would be deleted to make the retrieve process as an atomic action.

    \begin{figure}[H]
        \centering \includegraphics[scale=0.45]{grafiken/stateRestore.png}
        \caption{State Diagram of MARS project retrieval process considering empty states}
        \label{fig:stateRestore}
    \end{figure}

    Figure \ref{fig:stateRestore} illustrates the transitions that can occur in the retrieval process. The state digram has a very similar procedure as for the
    archive (Figure \ref{fig:stateArchive}) as both execute their actions in the same order. 
    Also, marking of the resources is not done before the start process, in contrast to the archive because Data Coherency issues are not present as the archived
    data is store in a centralized storage i.e. Synology accessed only by archive service.   

\newpage
Figure \ref{fig:sequenceRestore} illustrates the sequence diagram for the retrieve process. The starting step is similar to the archive process where a
check is made if a process for the project is running or not. If a process is found running then the restore process will be denied. The first after a
successful job creation is to fetch the metadata from the archives. Using the metadata the corresponding files are uploaded. After the file upload 
they need some more time to be processed. Therefore, the retrieve
also waits for the completion for the processing of all files.  This is an important step as it is required for the files to be in the FINISHED state
in order to create an scenario and result configurations.  As mentioned in
\ref{subsec:restoreProb} the necessary attributes would be swapped using the new model. Similarly, the simulation plan would be then uploaded followed
by the simulation runs. Lastly, the restore for the simulation results will be made. The retrieve process will come to halt after a finished
status of restore is received. 

\begin{figure}[H]
    \centering \includegraphics[scale=0.5, angle=90, origin=c]{grafiken/sequenceRestore.png}
    \caption{Sequence Diagram for the restore process}
    \label{fig:sequenceRestore}
\end{figure}
    
