\section{API Design}
\label{section:APIDesign}
    The Archive service must expose its web API, so that an interaction between the its clients and the application can occur.
    The Representational State Transfer (REST) \cite[Chapter.~5]{REST} architectural approach is chosen to design its API. This approach is a 
    common approach to build a distributed system as it is technology independent. Therefore using this architecture, the system can later 
    evolve for any kind of system used for providing a broader layer of flexibility to the Archive service. 
    Also, the standardized aspect of RESTful service enables a software to create reusable elements \cite{RESTThesis}. A combination of HTTP with REST 
    to do a CRUD (Create, Read, Update, Delete) operation is preferred because HTTP is widely supported by most clients and programming languages (e.g. web browsers).
    The CRUD over HTTP consists of few uniform noun based interaction that can be executed by the client \cite[p.~13]{RESTThesis}. The
    Table \ref{table:archiveEndpoints} describes the API endpoint for archive, retrieve, and job status with a brief description.
    HTTP CRUD operations which are going to be implemented are described in Table \ref{table:curdHttp}. 

    \begin{table}[H]
        \centering
        \begin{tabular}{|p{2cm}|p{4cm}|p{7.5cm}|}
            \hline
                \textbf{HTTP Verb}  & \textbf{Description} & \textbf{Application}\\
            \hline
                POST & 
                Creates a new resources and dependent resources.
                & The POST request will be used to archive and retrieve the projects because new resources are being created for these requests.\\
            \hline
                GET & Reads the resource. & The GET request will be used to check the status of the archive and retrieve process. \\
            \hline
            PUT & Updates the resource. & The PUT request will be used to check the status of the archive and retrieve process. \\
            \hline
                DELETE & Deletes the resource. & The DELETE request will be used to delete a running archive or retrieve process. \\                
            \hline
        \end{tabular}
        \caption{CRUD interaction over HTTP in Archive service}
        \label{table:curdHttp}     
    \end{table}   
    
    \begin{table}[H]
        \centering
        \begin{tabular}{|p{6cm}|p{8cm}|}
            \hline
                \textbf{API Endpoint}&\textbf{Description}\\
            \hline
                archive/archiveProject/{{projectId}} & Archives a project given an id. This is a HTTP POST method.\\
            \hline
                retrieve/retrieveProject/{{projectId}} & Restores a project given an id. This is a HTTP POST method.\\
            \hline
                job/status/{{projectId}} & Gets the status of the archive or retrieve process, given an project id. This is a HTTP GET method.\\
            \hline
                job/status/{{jobId}} & Gets the status of the archive or retrieve process, given a job id. This is a HTTP GET method.\\
            \hline
                delete/project/{{projectId}} & Deletes the archived project from the synology drive, given a job id. This is a HTTP DELETE method.\\
            \hline
        \end{tabular}
        \caption{API Endpoints description for Archive service}
        \label{table:archiveEndpoints}     
    \end{table}  

    The API end points are designed considering the fact that more functionality could be added to the archive service without big changes needed in the client. 
    For example, the endpoint \textbf{"archive/archiveProject/{{projectId}}"} is designed thinking an archive could be also extended for other resources except the project. When
    the Archive service would like to support, archiving of only the simulation results, the endpoint for it would be \textbf{"archive/archiveSimulationResult/{{SimulationId}}"}.
    Hence, it would make it more flexible for the client to add the functionality without much effort. Also, to avoid multiple API call and
    increase the performance of the server, the get status (Table \ref{table:archiveEndpoints}) endpoint combines vital information needed for the client in one 
    request (See Listing \ref{lst:status}). 
    
\begin{lstlisting}[caption={Sucessful GET request for a archive status}, language=json,firstnumber=1, captionpos=b, label={lst:status}]
{
    "status": "PROCESSING",
    "projectId": "70C961b7-89bf-4bd5-bf61-31b6a17a15d9",
    "error": "NO ERROR",
    "lastUpdate": "2018-06-17T10:05:50.216Z",
    "archiveName": "NONE",
    "markSessionId": "AK5961b7-89bf-4bd5-bf61-31b67a15d88",
    "jobId": "855961b7-89bf-4bd5-bf61-31b6a17a15d3",
    "currentProcess": "Archive"
}
\end{lstlisting}







    
