\section{API Design}
    The Archive service must expose its web APIs to its client so that an interaction between the application can occur.
    The Representational State Transfer (REST) \cite[Chapter.~5]{REST} architectural approach is chosen for this thesis
    to design its API. This approach is a 
    common approach to build a distributed system as it is independent of any system. Therefore using this architecture the system can later 
    evolve for any kind of system which could be used in the future providing a broader layer of flexibility to the Archive service. 
    Also, the standardized aspects of RESTful service enables a software to create a reusable elements \cite{RESTThesis}. A combination of HTTP with REST to carry out
    CURD operations is more preferred since HTTP is widely supported by most clients and programming languages (e.g. web browsers).
    The CURD over HTTP consists of few uniform noun based interaction that can be executed by the client \cite[p.~13]{RESTThesis}. The
    HTTP CURD operations which are going to be implemented are described in Table \ref{table:curdHttp}.

    \begin{table}[h!]
        \centering
        \begin{tabular}{|p{2cm}|p{4cm}|p{7cm}|}
            \hline
                \textbf{HTTP Verb}  & \textbf{Description} & \textbf{Application}\\
            \hline
                POST & 
                Creates a new resources and dependent resources.
                & The POST request will be used to archive and retrieve the projects because new resources are being created for these requests.\\
            \hline
                GET & Reads the resource. & The GET request will be used to check the status of the archive and retrieve process. \\
            \hline
                DELETE & Deletes the resource. & The DELETE request will be used to delete a running archive or retrieve process. \\                
            \hline
        \end{tabular}
        \caption{CURD interaction over HTTP in Archive service}
        \label{table:curdHttp}     
    \end{table}   
    
    Table \ref{table:archiveEndpoints} mentions the API endpoint for archive, retrieve and job status with brief description.
    \begin{table}[H]
        \centering
        \begin{tabular}{|p{6cm}|}
            \hline
                \textbf{API Endpoint}\\
            \hline
                archive/archiveProject/{{projectId}} \\
            \hline
                retrieve/retrieveProject/{{projectId}} \\
            \hline
                job/status/{{projectId}} \\
            \hline
                job/status/{{jobId}} \\
            \hline
        \end{tabular}
        \caption{API Endpoints description for Archive service}
        \label{table:archiveEndpoints}     
    \end{table}   


    
